La partie scénario du projet a pour rôle global de gérer la réaction du systéme aux actions des joueurs. Le scénario ne gardera pas d'information sur l'état du systéme et demandera l'état actuel du systéme à la simulation chaque fois que ce sera nécésaire. Plus précisement le scénario aura le rôle suivant :
\begin{itemize}
\item Client: Informer l'UI des actions que peuvent réaliser les terroristes et le chef de la police. Par exemple interdire le deplacement d'un joueur dans un mur, ou informer un joueur qu'il est dans une zone ou il peut planter une bombe. 
\item Client: Recevoir les actions éxécutés par les joueurs, appliquer les changements locaux de ces actions puis les transmettre à la simulation.
\item Client: Recevoir les changements de l'état impactant les actions disponibles pour le joueur. Pour communiquer avec l'état local du côtès du client en évitant de demander trop souvent des données on utilise un systéme d'event. Le scénario envoie un nouvel event à l'état local et lorsque l'un de ces event a lieu, l'état local en informe le scénario.
\item Serveur: Recevoir du réseau les actions des joueurs et les appliquées chez la simulation. Par exemple: l'explosion d'une bombe augmente le niveaux de panique et de destruction d'une zone.
\item Serveur: Appliquer les régles globales sur la ville. Par exemple la proximité d'un terroriste avec un batiment augmente automatiquement ça connaissance de celui-ci. On utilise ici aussi une communication par event comme chez le client.
\end{itemize}
