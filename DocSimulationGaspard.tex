\documentclass[a4paper]{article}

\usepackage[french]{babel}
\usepackage[utf8x]{inputenc}
\usepackage[T1]{fontenc}
\usepackage{amsmath}
\usepackage{graphicx}

\begin{document}

\section*{Fonctionnement de la partie Simulation.*}

La variable map contient un objet de type géography, qui est en gros une matrice de Tile.

\subsection*{Anxiété :}
\paragraph{} Les cases contiennent un float de 0 à 100 qui correspond à l'anxiété localisée dans la case. 0 est le plus bas 100 est le plus haut et 10 est le niveau de base.
Chaque seconde, modulo les évenements supplémentaires qui modifient directement les cases, on a la formule de lissage suivante :\\
$Anx'[i,j]=$\\
$\sqrt{(\frac{(Anx[i-1,j])^2+(Anx[i,j-1])^2+(Anx[i+1,j])^2+(Anx[i,j+1])^2+\frac{(Anx[i-1,j-1])^2}{2}+\frac{(Anx[i+1,j+1])^2}{2}+\frac{(Anx[i-1,j+1])^2}{2}+\frac{(Anx[i+1,j-1])^2}{2}+2*(Anx[i,j])^2}{8}}$
(La formule dépasse volontairement)\\
Où Anx' est l'anxiété à l'instant (t+1) et Anx est l'anxiété à l'instant t.\\
Les cases adjacentes sont comptées avec le coefficient 1, la propre case avec le coefficient 2 et les cases en diagonale sont comptées avec le coefficient $\frac{1}{2}$. Le tout est pris dans une sorte de norme deux pour faire remonter un peu plus les extremes.

\subsection*{Population :}
Chaque Tile contient un niveau de population entre 0 et 100 exprimé en float: 0 le minimum, 100 le maximum et on veut une moyenne de 50 environ sur la carte.
Chaque seconde, on crée/supprime des npcs de la façon suivante :
\begin{itemize}
\item Un npc a $\frac{Pop[i,j]}{10}\%$ de chance d'être créé dans la case [i][j].
\item Un npc a $max(0,\frac{50-Pop[i,j]}{10})\%$ de chance de disparaitre dans la case [i][j].
\end{itemize}

\subsection*{Déplacement des npcs :}
A chaque tour de boucle, on appelle la méthode updatePosition des npcs.

\subsection*{Croisement des npcs :}
(A venir) Les npcs partagent leur peur quand ils se croisent.

\end{document}
